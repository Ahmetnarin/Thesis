Unser Alltag und Geschäftsleben wird immer mehr von machine Learning (ML) beeinflusst. Beispiele für den Einsatz begegnen uns im täglichen Leben bereits überall. Einige davon sind Gesichts-und Spracherkennung, personalisierte Produktempfehlungen bei Amazon, Börsenprognose für Spekulanten etc. Aber auch in sicherheitskritischen Systeme wie autonomes Fahren wird das ML-Verfahren eingesetzt. Das Ziel von dieser Arbeit ist, dass die Datenverarbeitung von solchen sicherheitskritischen Anwendungen beschleunigt werden, indem die Hardware in der Lage sind, um parallele Rechenperationen auszuführen. Um dieses Ziel zu erreichen wird die Systolic Array Architektur implementiert. Somit wird die Rechenleistung von Embedded Systems und IoT-Geräten erhöht. Dabei muss man den Stromverbrauch nicht ignorieren. Der Stromverbrauch von solchen Geräten spielt genauso wichtige Rolle wie Beschleunigung von Hardware. Denn die superschnelle  aber stromverschwenderische Hardware sind in mobilen Geräten nicht einsetzbar.  


%\cleardoublepage

\chapter*{Urheberrecht}

ARM\TReg, AMBA\TReg, AXI\TTra, Cortex\TTra, TrustZone\TTra, SecurCore\TTra  , DSTREAM\TTra und weitere im Text erwähnte ARM-Produkte sowie die entsprechenden Logos sind Marken oder eingetragene Marken der Advanced RISC Machines Ltd.\par
\vspace{0.5cm}
Xilinx\TReg, Zynq\TTra und weitere im Text erwähnte Xilinx-Produkte sowie die entsprechenden Logos sind Marken oder eingetragene Marken der Xilinx Inc.\par
\vspace{0.5cm}